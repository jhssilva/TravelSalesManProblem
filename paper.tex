% This template has been tested with LLNCS DOCUMENT CLASS -- version 2.21 (12-Jan-2022)

% !TeX spellcheck = en-US
% LTeX: language=en-US
% !TeX encoding = utf8
% !TeX program = pdflatex
% !BIB program = bibtex
% -*- coding:utf-8 mod:LaTeX -*-

% "a4paper" enables:
%
%  - easy print out on DIN A4 paper size
%
% One can configure a4 vs. letter in the LaTeX installation. So it is configuration dependend, what the paper size will be.
% This option  present, because the current word template offered by Springer is DIN A4.
% We accept that DIN A4 cause WTFs at persons not used to A4 in USA.
%
% "runningheads" enables:
%
%  - page number on page 2 onwards
%  - title/authors on even/odd pages
%
% This is good for other readers to enable proper archiving among other papers and pointing to
% content. Even if the title page states the title, when printed and stored in a folder, when
% blindly opening the folder, one could hit not the title page, but an arbitrary page. Therefore,
% it is good to have title printed on the pages, too.
%
% To disable outputting page headers and footers, remove "runningheads"
\documentclass[runningheads,a4paper,english]{llncs}[2022/01/12]

% backticks (`) are rendered as such in verbatim environments.
% See following links for details:
%   - https://tex.stackexchange.com/a/341057/9075
%   - https://tex.stackexchange.com/a/47451/9075
%   - https://tex.stackexchange.com/a/166791/9075
\usepackage{upquote}

% Set English as language and allow to write hyphenated"=words
%
% Even though `american`, `english` and `USenglish` are synonyms for babel package (according to https://tex.stackexchange.com/questions/12775/babel-english-american-usenglish), the llncs document class is prepared to avoid the overriding of certain names (such as "Abstract." -> "Abstract" or "Fig." -> "Figure") when using `english`, but not when using the other 2.
% english has to go last to set it as default language
\usepackage[ngerman,main=english]{babel}
%
% Hint by http://tex.stackexchange.com/a/321066/9075 -> enable "= as dashes
\addto\extrasenglish{\languageshorthands{ngerman}\useshorthands{"}}
%
% Fix by https://tex.stackexchange.com/a/441701/9075
\usepackage{regexpatch}
\makeatletter
\edef\switcht@albion{%
  \relax\unexpanded\expandafter{\switcht@albion}%
}
\xpatchcmd*{\switcht@albion}{ \def}{\def}{}{}
\xpatchcmd{\switcht@albion}{\relax}{}{}{}
\edef\switcht@deutsch{%
  \relax\unexpanded\expandafter{\switcht@deutsch}%
}
\xpatchcmd*{\switcht@deutsch}{ \def}{\def}{}{}
\xpatchcmd{\switcht@deutsch}{\relax}{}{}{}
\edef\switcht@francais{%
  \relax\unexpanded\expandafter{\switcht@francais}%
}
\xpatchcmd*{\switcht@francais}{ \def}{\def}{}{}
\xpatchcmd{\switcht@francais}{\relax}{}{}{}
\makeatother

% Links behave as they should. Enables "\url{...}" for URL typesettings.
% Allow URL breaks also at a hyphen, even though it might be confusing: Is the "-" part of the address or just a hyphen?
% See https://tex.stackexchange.com/a/3034/9075.
\usepackage[hyphens]{url}

% When activated, use text font as url font, not the monospaced one.
% For all options see https://tex.stackexchange.com/a/261435/9075.
% \urlstyle{same}

% Improve wrapping of URLs - hint by http://tex.stackexchange.com/a/10419/9075
\makeatletter
\g@addto@macro{\UrlBreaks}{\UrlOrds}
\makeatother

% nicer // - solution by http://tex.stackexchange.com/a/98470/9075
% DO NOT ACTIVATE -> prevents line breaks
%\makeatletter
%\def\Url@twoslashes{\mathchar`\/\@ifnextchar/{\kern-.2em}{}}
%\g@addto@macro\UrlSpecials{\do\/{\Url@twoslashes}}
%\makeatother

% This is the modern package for "Computer Modern".
% In case this gets activated, one has to switch from cmap package to glyphtounicode (in the case of pdflatex)
\usepackage[%
    rm={oldstyle=false,proportional=true},%
    sf={oldstyle=false,proportional=true},%
    % By using 'variable=true' the monospaced font can be used as variable font (with differents widths per letter)
    % However, this makes listings look ugly.
    tt={oldstyle=false,proportional=true,variable=false},%
    qt=false%
]{cfr-lm}

% Has to be loaded AFTER any font packages. See https://tex.stackexchange.com/a/2869/9075.
\usepackage[T1]{fontenc}

% Character protrusion and font expansion. See http://www.ctan.org/tex-archive/macros/latex/contrib/microtype/

\usepackage[
  babel=true, % Enable language-specific kerning. Take language-settings from the languge of the current document (see Section 6 of microtype.pdf)
  expansion=alltext,
  protrusion=alltext-nott, % Ensure that at listings, there is no change at the margin of the listing
  final % Always enable microtype, even if in draft mode. This helps finding bad boxes quickly.
        % In the standard configuration, this template is always in the final mode, so this option only makes a difference if "pros" use the draft mode
]{microtype}

% \texttt{test -- test} keeps the "--" as "--" (and does not convert it to an en dash)
\DisableLigatures{encoding = T1, family = tt* }

%\DeclareMicrotypeSet*[tracking]{my}{ font = */*/*/sc/* }%
%\SetTracking{ encoding = *, shape = sc }{ 45 }
% Source: http://homepage.ruhr-uni-bochum.de/Georg.Verweyen/pakete.html
% Deactiviated, because does not look good

\usepackage{graphicx}

% Diagonal lines in a table - http://tex.stackexchange.com/questions/17745/diagonal-lines-in-table-cell
% Slashbox is not available in texlive (due to licensing) and also gives bad results. Thus, we use diagbox
\usepackage{diagbox}

\usepackage{xcolor}

% Code Listings
\usepackage{listings}
\usepackage{float}

\definecolor{eclipseStrings}{RGB}{42,0.0,255}
\definecolor{eclipseKeywords}{RGB}{127,0,85}
\colorlet{numb}{magenta!60!black}

% JSON definition
% Source: https://tex.stackexchange.com/a/433961/9075

\lstdefinelanguage{json}{
    basicstyle=\normalfont\ttfamily,
    commentstyle=\color{eclipseStrings}, % style of comment
    stringstyle=\color{eclipseKeywords}, % style of strings
    numbers=left,
    numberstyle=\scriptsize,
    stepnumber=1,
    numbersep=8pt,
    showstringspaces=false,
    breaklines=true,
    frame=lines,
    % backgroundcolor=\color{gray}, %only if you like
    string=[s]{"}{"},
    comment=[l]{:\ "},
    morecomment=[l]{:"},
    literate=
        *{0}{{{\color{numb}0}}}{1}
         {1}{{{\color{numb}1}}}{1}
         {2}{{{\color{numb}2}}}{1}
         {3}{{{\color{numb}3}}}{1}
         {4}{{{\color{numb}4}}}{1}
         {5}{{{\color{numb}5}}}{1}
         {6}{{{\color{numb}6}}}{1}
         {7}{{{\color{numb}7}}}{1}
         {8}{{{\color{numb}8}}}{1}
         {9}{{{\color{numb}9}}}{1}
}

\lstset{
  % everything between (* *) is a latex command
  escapeinside={(*}{*)},
  %
  language=json,
  %
  showstringspaces=false,
  %
  extendedchars=true,
  %
  basicstyle=\footnotesize\ttfamily,
  %
  commentstyle=\slshape,
  %
  % default: \rmfamily
  stringstyle=\ttfamily,
  %
  breaklines=true,
  %
  breakatwhitespace=true,
  %
  % alternative: fixed
  columns=flexible,
  %
  numbers=left,
  %
  numberstyle=\tiny,
  %
  basewidth=.5em,
  %
  xleftmargin=.5cm,
  %
  % aboveskip=0mm,
  %
  % belowskip=0mm,
  %
  captionpos=b
}

% Enable Umlauts when using \lstinputputlisting.
% See https://stackoverflow.com/a/29260603/873282 für details.
% listingsutf8 did not work in June 2020.
\lstset{literate=
  {á}{{\'a}}1 {é}{{\'e}}1 {í}{{\'i}}1 {ó}{{\'o}}1 {ú}{{\'u}}1
  {Á}{{\'A}}1 {É}{{\'E}}1 {Í}{{\'I}}1 {Ó}{{\'O}}1 {Ú}{{\'U}}1
  {à}{{\`a}}1 {è}{{\`e}}1 {ì}{{\`i}}1 {ò}{{\`o}}1 {ù}{{\`u}}1
  {À}{{\`A}}1 {È}{{\'E}}1 {Ì}{{\`I}}1 {Ò}{{\`O}}1 {Ù}{{\`U}}1
  {ä}{{\"a}}1 {ë}{{\"e}}1 {ï}{{\"i}}1 {ö}{{\"o}}1 {ü}{{\"u}}1
  {Ä}{{\"A}}1 {Ë}{{\"E}}1 {Ï}{{\"I}}1 {Ö}{{\"O}}1 {Ü}{{\"U}}1
  {â}{{\^a}}1 {ê}{{\^e}}1 {î}{{\^i}}1 {ô}{{\^o}}1 {û}{{\^u}}1
  {Â}{{\^A}}1 {Ê}{{\^E}}1 {Î}{{\^I}}1 {Ô}{{\^O}}1 {Û}{{\^U}}1
  {Ã}{{\~A}}1 {ã}{{\~a}}1 {Õ}{{\~O}}1 {õ}{{\~o}}1
  {œ}{{\oe}}1 {Œ}{{\OE}}1 {æ}{{\ae}}1 {Æ}{{\AE}}1 {ß}{{\ss}}1
  {ű}{{\H{u}}}1 {Ű}{{\H{U}}}1 {ő}{{\H{o}}}1 {Ő}{{\H{O}}}1
  {ç}{{\c c}}1 {Ç}{{\c C}}1 {ø}{{\o}}1 {å}{{\r a}}1 {Å}{{\r A}}1
}

% For easy quotations: \enquote{text}
% This package is very smart when nesting is applied, otherwise textcmds (see below) provides a shorter command
\usepackage[autostyle=true]{csquotes}

% Enable using "`quote"' - see https://tex.stackexchange.com/a/150954/9075
\defineshorthand{"`}{\openautoquote}
\defineshorthand{"'}{\closeautoquote}

% Nicer tables (\toprule, \midrule, \bottomrule)
\usepackage{booktabs}

% Extended enumerate, such as \begin{compactenum}
\usepackage{paralist}

% Bibliopgraphy enhancements
%  - enable \cite[prenote][]{ref}
%  - enable \cite{ref1,ref2}
% Alternative: \usepackage{cite}, which enables \cite{ref1, ref2} only (otherwise: Error message: "White space in argument")

% Doc: http://texdoc.net/natbib
\usepackage[%
  square,        % for square brackets
  comma,         % use commas as separators
  numbers,       % for numerical citations;
  %sort           % orders multiple citations into the sequence in which they appear in the list of references;
  sort&compress % as sort but in addition multiple numerical citations
                  % are compressed if possible (as 3-6, 15);
]{natbib}

% In the bibliography, references have to be formatted as 1., 2., ... not [1], [2], ...
\renewcommand{\bibnumfmt}[1]{#1.}

% Enable hyperlinked author names in the case of \citet
% Source: https://tex.stackexchange.com/a/76075/9075
\usepackage{etoolbox}
\makeatletter
\patchcmd{\NAT@test}{\else \NAT@nm}{\else \NAT@hyper@{\NAT@nm}}{}{}
\makeatother

% Prepare more space-saving rendering of the bibliography
% Source: https://tex.stackexchange.com/a/280936/9075
\SetExpansion
[ context = sloppy,
  stretch = 30,
  shrink = 60,
  step = 5 ]
{ encoding = {OT1,T1,TS1} }
{ }

% Put figures aside a text
\usepackage[rflt]{floatflt}

% Enable nice comments
\usepackage{pdfcomment}

\newcommand{\commentontext}[2]{\colorbox{yellow!60}{#1}\pdfcomment[color={0.234 0.867 0.211},hoffset=-6pt,voffset=10pt,opacity=0.5]{#2}}
\newcommand{\commentatside}[1]{\pdfcomment[color={0.045 0.278 0.643},icon=Note]{#1}}

% Compatibality with packages todo, easy-todo, todonotes
\newcommand{\todo}[1]{\commentatside{#1}}

% Compatiblity with package fixmetodonotes
\newcommand{\TODO}[1]{\commentatside{#1}}

% Put footnotes below floats
% Source: https://tex.stackexchange.com/a/32993/9075
\usepackage{stfloats}
\fnbelowfloat

\usepackage[group-minimum-digits=4,per-mode=fraction]{siunitx}
\addto\extrasgerman{\sisetup{locale = DE}}

% Enable that parameters of \cref{}, \ref{}, \cite{}, ... are linked so that a reader can click on the number an jump to the target in the document
\usepackage{hyperref}

% Enable hyperref without colors and without bookmarks
\hypersetup{
  hidelinks,
  colorlinks=true,
  allcolors=black,
  pdfstartview=Fit,
  breaklinks=true
}

% Enable correct jumping to figures when referencing
\usepackage[all]{hypcap}

\usepackage[caption=false,font=footnotesize]{subfig}

\usepackage{mindflow}

% Extensions for references inside the document (\cref{fig:sample}, ...)
% Enable usage \cref{...} and \Cref{...} instead of \ref: Type of reference included in the link
% That means, "Figure 5" is a full link instead of just "5".
\usepackage[capitalise,nameinlink]{cleveref}

\crefname{section}{Sect.}{Sect.}
\Crefname{section}{Section}{Sections}
\crefname{listing}{List.}{List.}
\crefname{listing}{Listing}{Listings}
\Crefname{listing}{Listing}{Listings}
\crefname{lstlisting}{Listing}{Listings}
\Crefname{lstlisting}{Listing}{Listings}

\usepackage{lipsum}

% For demonstration purposes only
% These packages can be removed when all examples have been deleted
\usepackage[math]{blindtext}
\usepackage{mwe}
\usepackage[realmainfile]{currfile}
\usepackage{tcolorbox}
\tcbuselibrary{listings}

%introduce \powerset - hint by http://matheplanet.com/matheplanet/nuke/html/viewtopic.php?topic=136492&post_id=997377
\DeclareFontFamily{U}{MnSymbolC}{}
\DeclareSymbolFont{MnSyC}{U}{MnSymbolC}{m}{n}
\DeclareFontShape{U}{MnSymbolC}{m}{n}{
  <-6>    MnSymbolC5
  <6-7>   MnSymbolC6
  <7-8>   MnSymbolC7
  <8-9>   MnSymbolC8
  <9-10>  MnSymbolC9
  <10-12> MnSymbolC10
  <12->   MnSymbolC12%
}{}
\DeclareMathSymbol{\powerset}{\mathord}{MnSyC}{180}

\usepackage{xspace}
%\newcommand{\eg}{e.\,g.\xspace}
%\newcommand{\ie}{i.\,e.\xspace}
\newcommand{\eg}{e.\,g.,\ }
\newcommand{\ie}{i.\,e.,\ }

% Enable hyphenation at other places as the dash.
% Example: applicaiton\hydash specific
\makeatletter
\newcommand{\hydash}{\penalty\@M-\hskip\z@skip}
% Definition of "= taken from http://mirror.ctan.org/macros/latex/contrib/babel-contrib/german/ngermanb.dtx
\makeatother

% Add manual adapted hyphenation of English words
% See https://ctan.org/pkg/hyphenex and https://tex.stackexchange.com/a/22892/9075 for details
% Does not work on MiKTeX, therefore disabled - issue reported at https://github.com/MiKTeX/miktex-packaging/issues/271
% \input{ushyphex}

% correct bad hyphenation here
\hyphenation{op-tical net-works semi-conduc-tor}

% Add copyright
%
% This is recommended if you intend to send the version to colleagues
% See https://ctan.org/pkg/llncsconf for details
\iffalse
  % state: intended | submitted | llncs
  % you can add "crop" if the paper should be cropped to the format Springer is publishing
  \usepackage[intended]{llncsconf}

  \conference{name of the conference}

  % in case of "proceedings" (final version!)
  % example: \llncs{Anonymous et al. (eds). \emph{Proceedings of the International Conference on \LaTeX-Hacks}, LNCS~42. Some Publisher, 2016.}{0042}
  % 0042 denotes an example start page
  \llncs{book editors and title}{0042}
\fi

\graphicspath{ {./images/} }

% Enable copy and paste of text from the PDF
% Only required for pdflatex. It "just works" in the case of lualatex.
% Alternative: cmap or mmap package
% mmap enables mathematical symbols, but does not work with the newtx font set
% See: https://tex.stackexchange.com/a/64457/9075
% Other solutions outlined at http://goemonx.blogspot.de/2012/01/pdflatex-ligaturen-und-copynpaste.html and http://tex.stackexchange.com/questions/4397/make-ligatures-in-linux-libertine-copyable-and-searchable
% Trouble shooting outlined at https://tex.stackexchange.com/a/100618/9075
%
% According to https://tex.stackexchange.com/q/451235/9075 this is the way to go
\input glyphtounicode
\pdfgentounicode=1

\begin{document}

\title{Traveling Salesman Problem}

% Single insitute
\author{Hugo Silva}

\institute{University of Évora, Portugal\\}

\maketitle

\begin{abstract}
  The present paper describes the solution of the Traveling Salesman Problem using Constraint Programming.
  The problem is modeled as a Constraint Satisfaction Problem and solved using the Choco Solver.
  The results obtained are displayed and discussed in the present paper.


\keywords{CSP \and Traveling Salesman Problem \and Constraint Programming}
\end{abstract}


\section{Introduction}
\label{sec:introduction}

The present paper starts with the a brief description of the problem (\cref{sec:problem}).
It is followed by a brief summary of the state of the art (\cref{sec:StateArt}), the formulation of the CSP (\cref{sec:CSP}), then a brief description of the tool used (\cref{sec:Framework}).\newline
Then it describes the model (\cref{sec:ModelDescription}) used, the results obtained (\cref{sec:Results}) and the discussion of the results (\cref{sec:ResultsDiscussion}).\newline
Finally, in the last section is drawn a conclusion for the present work (\cref{sec:Conclusion}).

\section{Problem}
\label{sec:problem}

The Traveling Salesman problem consists of a Salesman wanting to find the shortest path on his sales route. 
He only can visit each city once, then he must return to the starting city. \cite{BritannicaTSP}

\begin{figure}[H]
  \centering
  \includegraphics[width=0.66\linewidth]{image}
  \caption{Example figure with a route \cite{imageTSP}}
  \label{fig:ex:imageTSP}
\end{figure}

\section{State-of-the-art}
\label{sec:StateArt}
``The Travelling Salesman Problem (often called TSP) is a classic algorithmic problem in computer science''. \cite{imageTSP}
Is one of the most studied problems in computational intelligence and operations research. \cite{naturedInspiredTSP}

\section{CSP}
\label{sec:CSP}
A Constraint Satisfaction Problem (CSP) is defined by a set of variables, a set of variable domains, and a set of constraints on assignments to the variables. 
Each constraint specifies a set of one or more allowed variable assignments. \cite{CSPFormulation}\newline\newline
To formulate a CSP for a problem such as the Travelling salesman, 
it is necessary to visualize the CSP as a constraint graph. 
This means, that the nodes on the graph represent the CSP variables 
and the edges denote the constraints acting upon them. \cite{MSCSP}\newline\newline
For the Travelling Salesman Problem, the CSP is defined as:\newline
V = {Cities1, …, CitiesN};\newline
D = {1, …, N};\newline
C = \begin{itemize}
  \item The ending city must be equal to the starting city;
  \item $all_different(Cities1, …, CitiesN)$;
  \item $min (total distance)$.
\end{itemize}


\section{Framework}
\label{sec:Framework}
The framework chosen for the present article it’s the Choco-Solver. 
The Choco-Solver it’s an Open-Source Java library for constraint programming. \cite{chocoSolver}\newline
The framework was chosen by its simplicity, community, and easy accessibility.

\section{Model Description}
\label{sec:ModelDescription}
The Model is straightforward, based on the Constraint Satisfaction Problem (CSP) described previously in the present project on section \ref{sec:CSP}. \newline\newline
The Travelling Salesman Problem goal is to find the shortest path available between a range of cities while visiting all of them.\newline
It’s defined in the Model as an input of a matrix of distances, storing the distances between each pair of cities. \newline\newline
The Traveling Salesman model configuration for the example previously mentioned has the following characteristics:
\begin{itemize}
  \item Variables: 14;
  \item Constraints: 8;
  \item Building time: 0.090s;
  \item User-defined search strategy: yes;
  \item Complementary search strategy: no.
\end{itemize}
The following steps were taken to implement the Model:
\begin{enumerate}
  \item Define the Domains;
  \item Define the Variables;
  \item Define the Constraints;
  \item Define the Goal.
\end{enumerate}
The following code was implemented to program the model described previously: \newline
\begin{verbatim}
String cities[] = new String[] {"School", "Gym", "Planter's Farm",
 "Movies" , "Snell's Farm"};
int numberOfCities = cities.length;
int max = 20;
      
// matrix of distances
int[][] distancesBetweenCities = new int[][] {
        { 0, 10, -1, 10, -1},
        { 10, 0, 8, 15, -1},
        { -1, 8, 0, 12, 20},
        { 10, 15, 12, 0, 7},
        { -1, -1, 20, 7, 0}
};

// Creation of the Model
Model model = new Model("Traveling Salesman");

// Defining the VARIABLES
IntVar[] successiveCity = model.intVarArray("Successive City",
 numberOfCities, numberOfCities - 1);

// For each city, the distance to the succ visited one
IntVar[] distanceBetweenVisitedSuccessiveCity = model.intVarArray(
    "Distance Between Visited Successive City",
    numberOfCities,
    0,
    max
    );



// Total distance of the route
IntVar totalDistance = model.intVar(
    "Total distance",
    0,
    max * numberOfCities
    );

// Defining the CONSTRAINTS
for (int i = 0; i < numberOfCities; i++) {
    Tuples tuples = new Tuples(true);
    for (int j = 0; j < numberOfCities; j++) {
        if (j != i)
            tuples.add(j, distancesBetweenCities[i][j]);
    }

    model.table(
      successiveCity[i],
      distanceBetweenVisitedSuccessiveCity[i],
      tuples
      ).post();
}

// Defining a single circuit of size number of cities
//, visiting all cities
model.subCircuit(
    successiveCity,
    0, 
    model.intVar(numberOfCities)
    ).post();

// Defining the total distance
model.sum(distanceBetweenVisitedSuccessiveCity, "=", totalDistance).post();

// Set the Objective of the Model. Minimizing the total distance
model.setObjective(Model.MINIMIZE, totalDistance);

Solver solver = model.getSolver();
solver.setSearch(
        Search.intVarSearch(
                new MaxRegret(),
                new IntDomainMin(),
                distanceBetweenVisitedSuccessiveCity));
\end{verbatim}

\newpage
\section{Results}
\label{sec:Results}
In the present project, it was replicated the problem presented on the image \ref{fig:ex:imageTSP}. It was chosen a small set of values and a small problem, so we can observe and study the problem in a simple way. \newline \newline The results were the following:\newline
\textbf{ Number of Solutions:} 1
\begin{itemize}
  \item Building time: 0.090s;
  \item Resolution time: 0.024s;
  \item Nodes: 2 (82.8 n/s);
  \item Backtracks: 0;
  \item Backjumps: 0;
  \item Fails: 0;
  \item Restarts: 0;
  \item Path: School -> Gym -> Planter's Farm -> Snell's Farm -> Movies -> School;
  \item Total distance:  55 meters.
\end{itemize}
There was also a different test done with more input data. This test was done to have a more complex set of results and to see how the model performs.\newline
The test was done with different input values mapping 12 cities in Portugal. The cities are the following:
\begin{itemize}
  \item Barcelos;
  \item Povoa de Varzim;
  \item Vila do Conde;
  \item Porto;
  \item Braga;
  \item Aveiro;
  \item Famalicao;
  \item Guimaraes;
  \item Coimbra;
  \item Lisboa;
  \item Evora;
  \item Algarve.
\end{itemize}

For this example, the model had the following characteristics:
\begin{itemize}
  \item Variables: 28;
  \item Constraints: 15;
  \item Building time: 0.082s
\end{itemize}\newpage
When executing the model with the distances between cities as an input, the result was the following:\newline
\textbf{Number Of Solutions:} 4\newline\par
\textbf{First Solution:}
\begin{itemize}
  \item Building time: 0.082s;
  \item Resolution time: 0.047s;
  \item Nodes: 11 (235.4 n/s);
  \item Backtracks: 0;
  \item Backjumps: 0;
  \item Fails: 0;
  \item Restarts: 0;
  \item Path: Barcelos -> Povoa de Varzim -> Vila do Conde -> Algarve -> Evora -> Lisboa -> Coimbra -> Aveiro -> Porto -> Guimaraes -> Braga -> Famalicao -> Barcelos;
  \item Total distance: 1442 KM.
\end{itemize}


\textbf{Second Solution:}
\begin{itemize}
  \item Building time: 0.082s;
  \item Resolution time: 0.051s;
  \item Nodes: 13 (256.3 n/s);
  \item Backtracks: 3;
  \item Backjumps: 0;
  \item Fails: 1;
  \item Restarts: 0;
  \item  Path: Barcelos -> Povoa de Varzim -> Vila do Conde -> Algarve -> Evora -> Lisboa -> Coimbra -> Aveiro -> Porto -> Famalicao -> Guimaraes -> Braga -> Barcelos;
  \item Total distance: 1440 KM.
\end{itemize}

\textbf{Third Solution:}
\begin{itemize}
  \item Building time: 0.082s;
  \item Resolution time: 0.055s;
  \item Nodes: 18 (324.6 n/s);
  \item Backtracks: 13;
  \item Backjumps: 0;
  \item Fails: 5;
  \item Restarts: 0;
  \item Path: Barcelos -> Povoa de Varzim -> Vila do Conde -> Porto -> Algarve -> Evora -> Lisboa -> Coimbra -> Aveiro -> Guimaraes -> Braga -> Famalicao -> Barcelos;
  \item Total distance: 1436 KM.
\end{itemize}

\newpage
\textbf{Fourth Solution:}
\begin{itemize}
  \item Building time: 0.082s;
  \item Resolution time: 0.058s;
  \item Nodes: 19 (326.1 n/s);
  \item Backtracks: 14;
  \item Backjumps: 0;
  \item Fails: 5;
  \item Restarts: 0;
  \item Path: Barcelos -> Povoa de Varzim -> Vila do Conde -> Porto -> Algarve -> Evora -> Lisboa -> Coimbra -> Aveiro -> Famalicao -> Guimaraes -> Braga -> Barcelos;
  \item Total distance: 1434 KM.
\end{itemize}

\newpage
\section{Results Discussion}
\label{sec:ResultsDiscussion}
The results were quite satisfactory. Looking into the first example, with the places as described in the \ref{fig:ex:imageTSP}, we can observe that the model found effectively the shortest path with the shortest distance of 55 meters. This is if we consider that the Salesman only can visit a city once. For this model and present work, was considered the definition of the Travel Salesman Problem has ``The problem is to find a path that visits each city \textbf{once}, returns to the starting city, and minimizes the distance traveled.'' \cite{BritannicaTSP} \newline
In a real-life situation, the Salesman would be able to return to a city, instead of going directly to another city. \newline

In this case, it exists a more effective path as the following:
\begin{itemize}
  \item Path: School -> Gym -> Planter's Farm -> Movies -> Snell’s Farm -> Movies -> School;
  \item Total Distance: 54 meters.
\end{itemize}

This is an example of how the algorithm could be improved upon, by removing the constraint of all being different.\newline
Regarding the second example with the Portuguese Cities, the results are very satisfactory. The model found 4 solutions, with the shortest path being 1434 KM with a very fast resolution time of 0.058s. It wasn’t expected 4 different solutions for a small set of input values, however, the model performed very well and found the shortest path as expected. \newline
He was able to provide different paths. For example, for the first solution mentioned to the last one, it has a difference of 8 KM (1442KM - 1434KM), this is quite a remarkable cost-efficient improvement.

\section{Conclusion}
\label{sec:Conclusion}
The present work was able to provide a solution to the Travel Salesman Problem using the Constraint Programming Paradigm. The model was able to find the shortest path for a set of input values. \newline
It is quite remarkable the efficiency of the model, with a very fast resolution time of 0.058s. \newline
It's fascinating to use Constrain Programming to solve this kind of problem, as it is a very efficient way to solve it. \newline
The model could be improved upon, by removing the constraint of all being different, and by adding more constraints to the model, for example, the salesman could have a maximum distance to travel, or a maximum number of cities to visit. \newline
I'm curious to see the model being applied to real life examples, with more 100 of points to visit. Then to observe the results, and the performance.
The present project was a very interesting and challenging project, and I'm very happy with the results obtained.


\renewcommand{\bibsection}{\section*{References}} % requried for natbib to have "References" printed and as section*, not chapter*
% Use natbib compatbile splncs04nat style.
% It does provide all features of splncs04.bst, but is developed in a clean way.
% Source: https://github.com/tpavlic/splncs04nat
\bibliographystyle{splncs04nat}
\begingroup
  \microtypecontext{expansion=sloppy}
  \small % ensure correct font size for the bibliography
  \bibliography{paper}
\endgroup

% Enfore empty line after bibliography
\ \\
%

\end{document}
